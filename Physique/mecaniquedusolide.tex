
\chapter{Mécanique du solide}
\section{Lois de la mécanique d'un système materiel}
\subsection{Modélisation d'un système matériel}
On peut modifier un système materiel de deux façons : 
\begin{itemize}
 \item[$\rightarrow$] Approche discrète : $m_t = \sum m_i$
 \item[$\rightarrow$] Approche continue : $m_t = \iiint \rho.dV$
\end{itemize}
Dans la suite de l'exposé, on utilise la notation discrète, mais la notation continue est totalement utilisable.
\subsection{Théorème du centre d'inertie ( ou de la résultante cinétique) }
\subsubsection{Quantité de mouvement d'un système}
\begin{de}
On appelle quantité de mouvement d'un système, notée $\overrightarrow{P}$ : 
$$\overrightarrow{P} = \sum m_i.\overrightarrow{v_i}$$
En introduisant le barycentre, ou centre de masse, noté G : 
$$m_t.\overrightarrow{OG} = \sum m_i.\overrightarrow{OM_i}$$
On obtient que : 
$$\overrightarrow{P} = m_t.\overrightarrow{V_G}$$
\end{de}
\begin{theo}
Par application de la seconde loi de Newton, on obtient que : 
$$\dfrac{d\overrightarrow{P}}{dt} = \sum\overrightarrow{F_{ext}}$$
Ceci constitue le théorème du centre d'inertie
\end{theo}
\subsection{Référentiel Barycentrique}
\begin{de}
Un référentiel barycentrique est un référentiel qui a pour origine le centre des masses du système, et qui est en translation uniforme par rapport à un référentiel galiléen.
\end{de}
\begin{de}
On note $X^*$ la grandeur X dans le référentiel barycentrique.
\end{de}
\begin{prop}
Dans un référentiel barycentrique, nous avons : 
$$\overrightarrow{p^*} = \overrightarrow{0}$$
\end{prop}
\subsection{Système ouvert}
Pour un système ouvert, on considére la masse à l'instant t, et la masse à l'instant t+dt du système, plus la masse éjecté durant dt. On peut donc définir dans ce cas un système fermé.
\section{Théorème du moment cinétique}
\subsection{Moment cinétique}
\begin{de}
Considérons le moment cinétique d'un point materiel O, dans le référentiel R, défini par : 
$$\overrightarrow{L_{O/R}} = \overrightarrow{OM}\wedge m.\overrightarrow{v}$$
On defini donc le moment cinétique pour un ensemble de point par : 
$$\overrightarrow{L_{O/R} = \sum.\overrightarrow{OM}\wedge m.\overrightarrow{v_i}}$$ 
\end{de}
\begin{prop}
On obtient la formule de changement de point du moment cinétique : 
$$\overrightarrow{L_{A/R}} = \overrightarrow{L_{O/R}} + \overrightarrow{AO}\wedge\overrightarrow{p}$$
Avec $\overrightarrow{p}$ la quantité de mouvement du système. On obtient donc que le moment cinétique et la quantité de mouvement constitue un torseur cinétique
\end{prop}
\subsection{Théorème de Koening}
En partant de la décomposition de la vitesse : 
$$\overrightarrow{v_i} = \overrightarrow{v_{G/R}} + \overrightarrow{v_i^*}$$
Avec $\overrightarrow{v_{G/R}}$ la vitesse du centre de masse G par rapport au référentiel R, et $\overrightarrow{v_i^*}$ la vitesse du point i dans le référentiel barycentrique. \\
On obtient que :
$$\overrightarrow{L_{O/R}} = m_t.\overrightarrow{OG}\wedge\overrightarrow{v_G} + \overrightarrow{L_O^*}$$
On décompose donc le moment cinétique en un moment du à un mouvement de translation, et un autre du a un mouvement de rotation.
\begin{prop}
Le moment cinétique, dans le référentiel barycentrique, ne dépend pas du point où on le calcul. On l'écrit donc :
$$\overrightarrow{L^*}$$
\end{prop}
\subsection{Théorème du moment cinétique en un point fixe}
Soit O un point fixe par rappor à R. On obtient que : 
$$\dfrac{d\overrightarrow{L_{O/R}}}{dt} = \sum \overrightarrow{OM}\wedge\overrightarrow{F_{ext}}$$
\subsection{Théorème du moment cinétique en un point mobile}
Soit A un point mobile par rapport à R. On obtient que : 
$$\dfrac{d\overrightarrow{L_{A/R}}}{dt} = \sum \overrightarrow{AM}\wedge\overrightarrow{F_{ext}} - \overrightarrow{v_A}\wedge\overrightarrow{p}$$
En particulier, si R est un référentiel galiléen et que le point mobile est G, le centre de masse, on obtient que :
$$\dfrac{d\overrightarrow{L_{G/R}}}{dt} = \sum \overrightarrow{GM}\wedge\overrightarrow{F_{ext}}$$
\subsection{Représentation torsorielle}
Nous avons les représentations suivantes : 
$$\mbox{ Torseur Cinématique } \begin{cases}
                                \overrightarrow{P} \\ \overrightarrow{L_O}
                               \end{cases}
$$
$$\mbox{ Torseur Dynamique} \begin{cases}
                                \overrightarrow{R} = \sum \overrightarrow{F_{ext}} \\ \overrightarrow{\mathcal{M}} = \sum \overrightarrow{OM}\wedge\overrightarrow{F_{ext}}
                               \end{cases}
$$
On observe que le torseur dynamique est la dérivé du torseur cinématique
\section{Théorème de l'énergie cinétique}
\subsection{Énergie cinétique}
\begin{de}
Par définition, pour un ensemble de point materiel, on obtient : 
$$E_c = \sum \dfrac{1}{2}.m_i.v_i^2$$
\end{de}
\subsection{Théorème de Koening}
De la façon que précédement, on obtient que : 
$$\overrightarrow{E_c} = \dfrac{1}{2}.m_t.V_g^2 + \overrightarrow{E_c^*}$$
\subsection{Théorèmes}
\begin{prop}
En partant de l'expression de l'énergie cinétique, on obtient que :
$$\Delta E_c = W_{ext} + W_{int}$$
Avec $W_{ext}$ le travail des forces exterieurs et $W_{int}$ le travail des forces interieurs. Si le système est un solide, donc indéformable, on obtient : 
$$\Delta E_c = W_{ext}$$
\end{prop}
\section{Cas du solide}
\begin{de}
On défini un solide par : 
$$\forall (A,B) \in \mbox{Objet}^2~ \parallel\overrightarrow{AB}\parallel = cte$$
C'est donc un objet indéformable
\end{de}
\subsection{Cinétique}
\subsubsection{Champs de vitesse}
Considérons un solide, en mouvement dans un référentiel R fixe. On défini un référentiel R', lié au solide. Soit C le centre d'un repère lié a R', et M un point du solide. On obtient : 
$$\overrightarrow{v}_M = \overrightarrow{v}_C + \overrightarrow{\Omega}\wedge\overrightarrow{CM}$$
Avec $\overrightarrow{\Omega}$ le vecteur rotation instantané
\subsection{Torseur cinématique}
\subsubsection{Solide possédant un point fixe}
Considérons que le solide possède un point fixe, notée C. D'après la relation précédente, on obtient que, pour tous point M du solide :
$$ \overrightarrow{v}_M = \overrightarrow{\Omega}\wedge\overrightarrow{CM}$$
En utilisant le théorème du moment cinétique, on obtient que :
$$I = \sum m_i.r_i^2$$
Avec I le moment d'inertie par rapport aux axes.
\subsubsection{Solide possédant un axe fixe}
On obtient que le vecteur rotation instantanée est porté par l'axe fixe, Oz par exemple. En appliquant le théorème du moment cinétique, on obtient que :
$$\overrightarrow{L}_C = I_{Oz}.\overrightarrow{\Omega} + \overrightarrow{L}_{C\bot}$$
Si z est un axe de symétrie, on obtient que : 
$$\overrightarrow{L}_C = I_{Oz}.\overrightarrow{\Omega}$$
Ce qui s'écrit aussi : 
$$\overrightarrow{L}_C = \overrightarrow{L}_{C_{//}}$$
\subsection{Moment d'inertie par rapport à un axe}
\subsubsection{Relation de Huygens}
Soit $\Delta$ et $\Delta'$ deux droites parralèle séparé d'une distance A. On obtient que :
$$I_{\Delta'} = I_{\Delta} + m_t.A^2$$
Ceci constitue la relation de Huygens.
\subsection{Energie cinétique}
Supposons que le solide possède un point fixe, noté C. On obtient, en partant de l'expression général de l'énergie cinétique, et du champs de vitesse, la relation suivante : 
$$E_c = \dfrac{1}{2}.\overrightarrow{L}_C.\overrightarrow{\Omega}$$
De plus, si on suppose qu'il existe un axe de rotation instantanée $\Delta$ parralèle à $\overrightarrow{\Omega}$, on obtient que :
$$E_c = \dfrac{1}{2}I_{\Delta}.\Omega^2$$
\section{Contact entre deux solides}
\subsection{Types de mouvement relatif}
Il existe trois types de mouvement relatif : 
\begin{itemize}
 \item[$\rightarrow$] Mouvement de Translation
 \item[$\rightarrow$] Mouvement de Rotation
 \item[$\rightarrow$] Mouvement de Roulement
\end{itemize}
\subsection{Vitesse de glissement}
On considère deux solides en contact. À un instant t donnée, on suppose que les points $I_1$, du solide 1, et $I_2$, du solide 2, sont en contact. On obtient l'expression de la vitesse de glissement, noté $\overrightarrow{v}_g$ : 
$$\overrightarrow{v}_{g_{2\rightarrow 1}} = \overrightarrow{v}_{I_2/R} - \overrightarrow{v}_{I_1/R}$$
\subsection{Lois de Coulomb pour le glissement}
Considérons un contact. La résultante $\overrightarrow{R}$ se décompose en deux forces : 
\begin{itemize}
 \item[$\rightarrow$] $\overrightarrow{F}$, la force de frottement
 \item[$\rightarrow$] $\overrightarrow{n}$, la réaction normale au support
\end{itemize}
\subsubsection{En l'absence de glissement}
En l'absence de glissement, on obtient que : 
$$\dfrac{\parallel\overrightarrow{F}\parallel}{\parallel\overrightarrow{n}\parallel}\leq f_0$$
Avec $f_0$ le coefficiant de frotement statique
\subsubsection{Avec glissement}
Si il y a glissement, on obtient que : 
$$\dfrac{\parallel\overrightarrow{F}\parallel}{\parallel\overrightarrow{n}\parallel} =  f$$
Avec f le coefficiant de frottement dynamique, $f\ll f_0$
