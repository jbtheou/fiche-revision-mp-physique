
\chapter{Ondes électromagnétiques "confinées" avec conditions aux limites}
\section{Réflexion d'une onde plane progressive monochromatique sur un plan conducteur parfait}
\begin{de}
On considère qu'un conduteur est parfait quand : 
$$\gamma\rightarrow +\infty$$
Ce qui revient donc à :
$$\overrightarrow{E} = \overrightarrow{0}$$
\end{de}
\subsection{Réflexion normale}
Considérons un champs électrique et un champs magnétique de la forme : 
$$\overrightarrow{E_i} = E_{0i}cos(\omega.t-kx)\overrightarrow{u_y}$$
$$\overrightarrow{B_i} = \dfrac{E_{0i}}{c}.cos(\omega.t - kx)\overrightarrow{u_z}$$
\subsubsection{Onde réflechie}
On obtient une onde réfléchi, avec un coefficiant de réflexion -1, c'est à dire que :
$$\dfrac{E_{Or}}{E_{0i}} = -1$$
Il n'y a donc pas d'onde transmise. On obtient donc comme expression pour les champs réfléchis :
$$\overrightarrow{E_r} = -E_{oi}cos(\omega.t+kx)\overrightarrow{u_y}$$
$$\overrightarrow{B_r} =\dfrac{-\overrightarrow{u_x}\wedge\overrightarrow{E_r}}{c} = \dfrac{E_{oi}}{c}.cos(\omega.t+kx)\overrightarrow{u_z}$$
On obtient donc des ondes planes progressive, mais de sens opposé.
\subsubsection{Onde stationnaire}
En étudiant maintenant le champs électrique totale, et le champs magnétique totale, c'est à dire : 
$$\overrightarrow{E} = \overrightarrow{E_i} + \overrightarrow{E_r}$$
$$\overrightarrow{B} = \overrightarrow{B_i} + \overrightarrow{B_r}$$
On obtient que ces champs ne sont pas des ondes progressives :
$$\overrightarrow{E} = 2.E_{Oi}.sin(kx)sin(\omega.t).\overrightarrow{u_y}$$
$$\overrightarrow{B} = \dfrac{2.{E_{Oi}}}{c}.cos(kx).cos(\omega.t)\overrightarrow{u_z}$$
On obtient que ces champs s'annule periodiquement selon x. On obtient donc, par analogie avec les cordes vibrante, des ventres et des noeuds. Le champs vibre donc sur place.
\subsubsection{Énergie transporté}
En calculant le vecteur de Poyting, on obtient que : 
$$<\overrightarrow{\pi}> = 0$$
Sur une periode, il n'y a donc pas de transporte d'énergie.
\subsubsection{Interface}
En considérant les rélations de discontinuité des champs a la frontière, on obtient qu'en $x = 0^-$, les charges oscille, avec : 
$$\overrightarrow{j_s} = \dfrac{2.E_{Oi}}{\mu_0.c}.cos(\omega.t)\overrightarrow{u_y}$$
On obtient donc que les charges oscille à la surface du conducteur
\subsubsection{Phénomène de résonnance}
Considérons deux conducteurs parralèle séparé par une distance L. Il faut que L vérifie, pour pouvoir observé un phénomène de résonnance, la relation suivante : 
$$L = p.\dfrac{\lambda}{2}$$
\subsection{Réflexion oblique}
Considérons une onde arrivant oblique par rapport à l'axe Ox, formant l'angle i avec cette axe. Elle se réflechie sur le conducteur parfait. Nous avons : 
$$\overrightarrow{E_i} = E_{Oi}.cos(\omega.t-\overrightarrow{k_i}.\overrightarrow{OM})\overrightarrow{u_z}$$
$$\overrightarrow{B_i} = \dfrac{\overrightarrow{k_i}}{k}\wedge\dfrac{\overrightarrow{E_i}}{c}$$
Avec : 
$$\overrightarrow{k_i} = k.(cos(i).\overrightarrow{u_x}+sin(i).\overrightarrow{u_y})$$
$$\overrightarrow{k_r} = k.(-cos(i).\overrightarrow{u_x}+sin(i).\overrightarrow{u_y})$$
On obtient donc par continuité, c'est à dire sachant que le champs en nul en x=0, que : 
$$\overrightarrow{E_r} = -E_{Oi}.cos(\omega.t - \overrightarrow{k_r}\wedge\overrightarrow{OM})\overrightarrow{u_z}$$
$$\overrightarrow{B_r} = \dfrac{\overrightarrow{k_r}}{k}\wedge\dfrac{\overrightarrow{E_r}}{c}$$
En développant, a l'aide des équations de Maxwell, on obtient les expressions des champs totaux $\overrightarrow{E}$ et $\overrightarrow{B}$.
\section{Propagation guidée entre deux plans parallèle}
Considérons deux plans parallèle, l'un défini par x=0, l'autre par x=a. Par discontinuité des champs, nous avons les relations de passage suivante en 0:
$$\overrightarrow{E}_{x = 0^+} = \dfrac{\sigma}{\varepsilon_0}.\overrightarrow{u_x}$$
$$\overrightarrow{B}_{x = 0^+} = \mu_0.\overrightarrow{j_s}\wedge\overrightarrow{u_x}$$
Nous allons rechercher des solutions de la forme : 
$$\overrightarrow{E} = Re[\overrightarrow{\underline{E}]}(x,y).e^{i(\omega.t - k.z)}$$
On recherche donc des solutions qui se propage selon l'axe des z.
\subsection{Champs électrique transverse normale aux plaques}
On recherche un champs électrique porté par $\overrightarrow{u_x}$. En utilisant les équations de Maxwell, on montre que si : 
$$k = \dfrac{\omega}{c}$$
Alors :
$$E(x,y) = E_0$$
est une solutions du problème.
\subsection{Champs électrique parallèle aux plaques}
On recherche un champs électrique cette fois ci porté par $\overrightarrow{u_y}$. 
\subsubsection{Équation de propagation}
À l'aide de l'équation de propagation, on montre que le champs vérifie l'équation différentielle suivante :
$$\dfrac{\partial^2 E(x)}{\partial x^2} = (k^2 - \dfrac{\omega^2}{c^2}).E(x)$$
On observe donc que le type de solution dépend du signe de : 
$$(k^2 - \dfrac{\omega^2}{c^2})$$
\subsubsection{Conditions aux limites}
Les conditions aux limites nous disent que E(x) s'annule deux fois en 0 et en a. Le cas hyperbolique est donc impossible dans la résolution de l'équation différentielle, on obtient donc une solution sinusoïdale : 
$$\overrightarrow{E} = E_0.sin(\dfrac{n.\pi.x}{a}.cos(\omega.t - kx)).\overrightarrow{u_y}$$
\subsubsection{Relation de dispersion}
D'après l'équation différentielle et les conditions aux limites, nous savons que : 
$$\dfrac{n.\pi}{a} = \sqrt{\dfrac{\omega^2}{c^2} - k^2}$$
Ceci implique donc que : 
$$f \geq \dfrac{n.c}{2.a}$$
Le choix de n n'est donc pas totalement libre.\\
De plus, d'après l'expression de la vitesse de phase : 
$$v_{\phi} = \dfrac{\omega}{k}$$
On obtient que cette vitesse dépend de la fréquence.
\subsubsection{Vitesse de groupe}
En développant l'expression de l'énergie, on montre qu'elle se déplace à une vitesse de groupe : 
$$v_g = \dfrac{c^2}{v_{\varphi}}$$
Asymptotiquement en fréquence, on obtient donc que la vitesse de groupe et la vitesse de phases tendent vers c.
\subsection{Interprétation en termesde superposition d'onde plane}
On peut interpréter :
$$\overrightarrow{E} = E_0.sin(\dfrac{n.\pi.x}{a}.cos(\omega.t - kx)).\overrightarrow{u_y}$$
Qui n'est pas une onde plane, comme la superposition de deux ondes planes. On obtient cette décomposition à l'aide d'un changement de variable : 
$$\overrightarrow{E} = \dfrac{1}{2}.E_0[cos(\omega.t - \overrightarrow{k_1}.\overrightarrow{OM}) - cos(\omega.t - \overrightarrow{k_2}.\overrightarrow{OM})]$$
On obtient donc que ça peut être interprété comme la superposition de deux ondes planes. En développant les expression de $\overrightarrow{k_1}$ et de $\overrightarrow{k_2}$, on reconnait la figure d'onde réfléchi, et en calculant le module de ces vecteurs, on obtient que l'onde se propage à la vitesse de la lumière.
