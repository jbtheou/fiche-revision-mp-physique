\chapter{Ondes électromagnétique}
\section{Solution des équations de propagations}
\subsection{Équation de propagation}
D'après les quatres équations de Maxwell, nous pouvons obtenir des équations de propagations. Par exemple, pour l'équation de propagation du champs électrique $\overrightarrow{E}$, nous faisons : 
$$\overrightarrow{rot}(\overrightarrow{rot}(\overrightarrow{E}))$$
Qui, d'après les équations de Maxwell donne : 
$$\overrightarrow{rot}(\overrightarrow{rot}(\overrightarrow{E})) = \overrightarrow{rot}(-\dfrac{d\overrightarrow{B}}{dt})$$
En développant à l'aide de l'analyse vectorielle, et en se placant dans le vide, c'est à dire avec : 
$$\rho = 0$$
$$\overrightarrow{j} = 0$$
On obtient l'équation de propagation : 
$\overrightarrow{\Delta\overrightarrow{E}} - \varepsilon_0.\mu_0.\dfrac{\partial^2 \overrightarrow{E}}{\partial t^2} = 0$
On obtient de même pour $\overrightarrow{B}$
\subsection{Ondes planes}
\begin{de}
Une onde plane est une solution de l'équation de propagation qui ne dépend que d'une variable d'espace en coordonnée cartésienne.
\end{de}
Par exemple (ceci reste valable dans toute cette fiche), soit f une onde plane ne dépendant que de x et de t : 
$$f(x,t)$$
L'équation de propagation devient : 
$$\dfrac{\partial ^2 f}{\partial x^2} - \dfrac{1}{c^2}.\dfrac{\partial^2 f}{\partial t^2} = 0$$
En posant $\alpha$ et $\beta$ défini par : 
$$\alpha = x - ct$$
$$\beta = x + ct$$
Et en développant, on obtient que la solution général est une combinaison linéaire de deux ondes progressive ( c'est à dire qui translate le motif), qui progresse dans des sens opposés
\subsection{Onde sphérique}
\begin{de}
En coordonnée sphérique, une onde sphérique est une solution de l'équation de propagation qui ne dépend que de r et t. 
\end{de}
En développant de façon analogue à précédement, mais cette fois en utilisant y=rf, on obtient que cette solution est une onde progression atténué en $\dfrac{1}{r}$.
\section{Onde électromagnétique plane progressive}
\begin{de}
Une onde plane progressive, f, est définie par : 
\begin{itemize}
 \item[$\rightarrow$] Plane : f(x,t)
 \item[$\rightarrow$] Progresssive : f(x-ct), c'est à dire que : $x_2 - x_1 = c.(t_2-t_1)$
\end{itemize}
\end{de}
\subsection{Les champs E et B sont transverse}
En utilisant les divergences de ces champs, on montre respectivement que : 
$$\overrightarrow{E}\bot \overrightarrow{u_x}$$
$$\overrightarrow{B} \bot \overrightarrow{u_x}$$
\subsection{Les champs sont orthogonaux}
En partant de l'expression du rotationel de $\overrightarrow{E}$, on montre que : 
$$\overrightarrow{B} \bot \overrightarrow{E}$$
En se placant dans l'hypothèse d'une onde plane progressive.\\
On montre plus particulièrement que :
$$\overrightarrow{B} = \dfrac{\overrightarrow{u_x} \wedge \overrightarrow{E}}{c}$$
\subsection{Force exercée sur une particule chargée par l'onde}
La force exercée est la force de Lorentz. En remplacant la composante magnétique par l'expression de $\overrightarrow{B}$ déterminé précédement, on obtient que dans le cadre de le mécanique classique, c'est à dire pour des vitesse négligable devant la vitesse de la lumière, la force de Lorentz se ramène dans ce cas à : 
$$\overrightarrow{f} \backsimeq q.\overrightarrow{E}$$
\subsection{Vecteur de Poynting}
Par définition : 
$$\overrightarrow{\pi} = \dfrac{\overrightarrow{E}\wedge\overrightarrow{B}}{\mu_0}$$
En développent à l'aide de l'expression de $\overrightarrow{B}$ déterminé précédement, et à l'aide de la densité d'énergie dans l'espace, u, défini par : 
$$u = \varepsilon_0.E^2$$
On montre que : 
$$\overrightarrow{\pi} = c.u.\overrightarrow{u_x}$$
On obtient donc que l'énergie se progage selon l'axe de propagation et en étudiant la puissance, on obtient que l'énergie se déplace à la vitesse de la lumière.
\section{Onde plane progressive monochromatique}
\begin{de}
On dit que f est une onde monochromatique si l'on peut l'écrire sous la forme : 
$$f = A.cos(k(x-ct))$$
\end{de}
\subsection{Propriétés}
\subsubsection{Celle d'une onde progressive}
Cette onde possède toutes les propriétés énoncé précédement, de part son caratère progressive.
\subsubsection{Double periodicité}
Posons : 
$$kc = \omega$$
De ce fait, on peut écrire f sous la forme : 
$$f = A.cos(k.x-\omega.t)$$
En fixant t, on montre que la fonction est periodique par rapport à x, on défini sa période $\lambda$, periode spatial, par : 
$$\lambda = \dfrac{2\pi}{k}$$
De même, en fixant x, on montre que la fonction est periodique par rapport à t, on défini sa période T, période temporel, par : 
$$T = \dfrac{2\pi}{\omega}$$
On peut donc écrire f sous la forme suivante, qui fait apparaitre la double periodicité : 
$$f = A.cos(2\pi.(\dfrac{x}{\lambda} - \dfrac{t}{T}))$$
\subsubsection{Vecteur d'onde}
\begin{de}
On appelle vecteur d'onde :
$$\overrightarrow{k} = k.\overrightarrow{u_x}$$
avec k nombres d'onde.\\
$$f = A.cos(\overrightarrow{k}.\overrightarrow{OM}-\omega.t)$$
\end{de}
\subsection{Polarisation}
\subsubsection{Représentation complexe des champs}
On peut écrire le champs électrique par exemple sous la forme : 
$$\overrightarrow{E} = Re(\underline{\overrightarrow{E}}.e^{i(\omega.t-k.x)})$$
On peut aussi l'écrire sous la forme : 
$$\overrightarrow{E} = E_{oy}.\overrightarrow{u_y} + E_{oz}.e^{i\varphi}\overrightarrow{u_z}$$
\subsubsection{Onde polarisée rectilignement}
Cette onde est polarisée rectiligement si $\varphi=0$. Dans ce cas, l'onde décrit une droite en vibrant.
\subsubsection{Onde polarisée elliptiquement}
Si $\varphi \neq 0$, alors l'onde décrit en vibrant un éllipse. 
