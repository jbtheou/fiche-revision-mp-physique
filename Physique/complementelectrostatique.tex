\chapter{Complément d'électrocinétique}
\section{Définitions}
\subsection{Intensité de courant}
\begin{de}
L'intensité est un débit de charge : 
$$i = \dfrac{dq}{dt}$$
\end{de}
\subsection{Vecteur densité surfacique de courant}
\begin{de}
On considère une surface $\Sigma$. On peut écrire : 
$$ i = \underset{\Sigma}\iint \overrightarrow{j}.\overrightarrow{n}.dS$$
Avec $\overrightarrow{j}$ le vecteur densité surfacique de courant.
\end{de}
Considérons un modèle composé d'un seul type de porteur de charge. \\
On obtient que : 
$$dq = \rho_m.\overrightarrow{v}.\overrightarrow{n}.\Sigma.dt$$
On peut identifier et on obtient que :
$$\overrightarrow{j} = \rho_m.\overrightarrow{v}$$
Avec $\rho_m$ la densité de charge mobile. Par un théorème de superposition, on obtient dans un modèle de multiple porteur de charge : 
$$\overrightarrow{j} = \sum \rho_{m,i}.\overrightarrow{v_i}$$
\subsection{Équation de continuité}
Considérons un conducteur fermé de surface $\Sigma$. Nous avons les équations suivantes :
$$i = \underset{\Sigma}\iint \overrightarrow{j}.\overrightarrow{n}.dS$$
$$i = \dfrac{-dQ}{dt}$$
$$Q = \iiint_v \rho.dv$$
En considérant que cette surface est invariente avec le temps, on obtient que : 
$$\dfrac{\partial \rho}{\partial t} + div(\overrightarrow{j}) = 0$$
Ceci est une équation locale, car toute ces composantes dépendent du point M. Cette équation est l'expression de la conservation de la charge.\\
On en déduit qu'en régime permenant, conscient que dans ce cas $\rho$ est une constante par rapport au temps : 
$$div(\overrightarrow{j})= 0$$
\section{Expression de la puissance reçu par un dipôle}
Considérons un dipôle contenant des charges mobiles. En discrétisant le problème, on obtient l'expression du travail :
$$\omega = \sum q_i.\overrightarrow{E}(m_i).\overrightarrow{v_i}.dt$$
Cette somme est a réaliser sur l'ensemble des charges mobiles. Dans le cas continue, on obtient : 
$$\omega = \underset{v}\iiint \rho_{mobile}.d\tau.\overrightarrow{E}(M)<\overrightarrow{v}>dt$$
D'ou l'expression de la puissance, sachant que : 
$$P = \dfrac{d\omega}{dt} = \underset{v}\iiint \rho_{mobile}.d\tau.\overrightarrow{E}(M)<\overrightarrow{v}>$$
D'ou : 
$$P = \underset{v}\iiint \overrightarrow{j}.\overrightarrow{E}(M).d\tau$$
Cette expression est l'expression de la puissance reçu par le dipôle.\\
Au finale, on obtient donc que : 
$$P=i*(V_A-V_B)$$
Avec A et B respectivement le point d'entre et le point de sortie du dipôle.
\section{Conducteur ohmique}
\subsection{Loi d'Ohm}
Soit la loi d'Ohm : 
$$U = R.i$$
Cette relation revient à dire que : 
$$\overrightarrow{j} = \sigma.\overrightarrow{E}$$
Avec $\sigma$ la conductivité.\\
La loi d'ohm ne s'applique que si $\rho_m$ est une constante indépendante de $\overrightarrow{E}$. En pratique, un dipôle vérifie la loi d'ohm dans un certain intervalle de valeur pour $\overrightarrow{E}$.
\subsection{Résistance électrique}
\begin{de}
Considérons un dipôle électrique. On suppose qu'il vérifie la loi d'ohm, donc : 
$$\overrightarrow{j}=\sigma.\overrightarrow{E}$$
De plus, nous avons les équations suivantes :
$$i = \underset{\Sigma}\iint \overrightarrow{j}.\overrightarrow{n}.dS$$
$$\overrightarrow{E} = -\overrightarrow{grad}(V)$$
On en déduit donc que : 
$$V_A - V_B = \int_A^B \overrightarrow{E}.\overrightarrow{dm}$$
On obtient donc que :
$$V_A - V_B = R.i$$
Avec R une constante, appelé résistance
\end{de}
\begin{prop}
On remarque des corrélations entre la capacité et la résistance d'un dipôle. On montre que dans une topologie proche, on as : 
$$R.C = \dfrac{\varepsilon_0}{\sigma}$$
\end{prop}
\subsection{Effet Joule}
Considérons un dipôle. Soit P la puissance reçu par ce dipôle. On obtient : 
$$P = \iiint \overrightarrow{j}.\overrightarrow{E}.d\tau$$
Or dans le cas d'un conducteur ohmique : 
$$\overrightarrow{j} = \sigma \overrightarrow{E}$$
Donc : 
$$P = \iiint \sigma.E^2.d\tau > 0$$
Donc un conducteur ohmique ne peut que consommé de l'énergie.
