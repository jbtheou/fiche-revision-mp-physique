\chapter{Forme locale de l'électrostatique}
\section{Champs électrique}
\subsection{Définitions}
\begin{de}
Soit E une application :
$$E : M \mapsto \overrightarrow{E}(M)$$
Si en un point M, on place une charge $q_0$, cette charge subit une force $\overrightarrow{f}$.\\
On défini donc le champs électrique par : 
$$\overrightarrow{E} = \dfrac{\overrightarrow{f}}{q_0}$$
On observe donc que le champs électrique est indépendant de la charge $q_0$. On obtient donc qu'en introduisant un champs, on se libère du point d'observation.
\end{de}
\subsubsection{Loi de Coulomb}
\begin{de}
Soit A($q_1$) un point matériel de charge $q_1$, B($q_2$), un point matériel de charge $q_2$. La force exercé de A sur B est donnée par, avec $r = \parallel\overrightarrow{AB}\parallel$ : 
$$\overrightarrow{f} = \dfrac{q_1.q_2}{4\pi\varepsilon_0.r^2}.\dfrac{\overrightarrow{AB}}{AB}$$
Avec : 
$$\dfrac{1}{4\pi\varepsilon_0} = 9.10^9$$.
On en déduit : 
$$\overrightarrow{E}(M) = \dfrac{1}{4\pi\varepsilon_0.r^2}.\dfrac{q.\overrightarrow{AB}}{AB}$$
\end{de}
\subsubsection{Calcul direct du champs}
Considérons un volume chargée.\\
Considérons une charge élémentaire dans ce volume, notée dq : 
$$dq = \rho.dv$$
Avec $\rho$ la densité volumique de charge de ce volume, et $dv$ le volume élémentaire contenant cette charge dq.\\
Soit M un point de l'espace, et P un point appartenant à dq.
On obtient : 
$$\overrightarrow{E}(M) = \dfrac{1}{4\pi\varepsilon_0}\underset{source}\iiint \dfrac{\rho.\overrightarrow{PM}}{PM^3}.dv$$
\section{Potentiel électrique}
\begin{de}
En partant de l'expression de la force déterminer dans le paragraphe à propos de la loi de coulomb : 
$$\overrightarrow{f} = \dfrac{q_1.q_2}{4\pi\varepsilon_0.r^2}.\dfrac{\overrightarrow{OM}}{OM}$$
On obtient que, avec : 
$$\delta \omega = \overrightarrow{f}.\overrightarrow{dm}$$
$$\overrightarrow{OM} = r.\overrightarrow{u_r}$$
$$\overrightarrow{dm}=dr.\overrightarrow{u_r}+r.d\theta.\overrightarrow{u_{\theta}}+r.sin(\theta)d\varphi.\overrightarrow{u_{\varphi}}$$
Donc : 
$$\delta \omega = -\dfrac{d}{dr}(\dfrac{q.q'.r}{4\pi\varepsilon_0})$$
On peut donc introduire une énergie potentielle (au sens mécanique du terme).
$$E_p = q'.\dfrac{q}{4.\pi.\varepsilon_0.r} = q'.V(m)$$
D'ou :
$$\overrightarrow{E}(M).\overrightarrow{dm} = -dV$$
On peut écrire cette relation sous la forme : 
$$\overrightarrow{E}(M) = -\overrightarrow{grad}V$$
En partant de l'expression du grandiant de V et de l'expression $\overrightarrow{dm}$ dans un système de coordonnée, on peut déterminer l'expression du champs électrique.
\end{de}
\subsection{Calcul direct du potentiel}
\begin{de}
Considérons un volume chargé, de densité volumique de charge $\rho$.\\
Considérons un volume élémentaire dv, avec P appartenant à dv, et M un point de l'espace :
$$V(M) = \underset{V charge}\iiint \dfrac{\rho.dv}{4\pi\varepsilon_0.PM}$$
\end{de}
\subsection{Condition pour qu'un champs de vecteurs soit un gradiant}
Soit $\overrightarrow{D}(M)$ un champs de vecteur.\\
Soit $\overrightarrow{rot}(\overrightarrow{D})$, le rotationnel de D, défini par : 
$$\overrightarrow{rot}\overrightarrow{D} = (\dfrac{\partial D_z}{\partial y}-\dfrac{\partial D_y}{\partial z};\dfrac{\partial D_x}{\partial z}-\dfrac{\partial D_z}{\partial x};\dfrac{\partial D_y}{\partial x}-\dfrac{\partial D_x}{\partial y})$$
Si $\overrightarrow{rot}(\overrightarrow{D}) = \overrightarrow{0}$, alors :
$$\exists g~ tq~ \overrightarrow{D}(M) = \overrightarrow{grad}(f)$$
\subsection{Lien entre direction de champs et surface equipotentielle}
\begin{prop}
$\overrightarrow{E}(M)$ sera parralèle à la surface équipotentielle V(M), avec $\overrightarrow{E}(M)$ dirige vers les potentiels décroissants
\end{prop}
\section{Théorème de Gauss}
\begin{enon}
Le flux du champ électrique à travers une surface fermé $\Sigma$ orienté vers l'extérieur est égale à la charge intérieur à cette surface sur la permitivité du vide, $\varepsilon_0$ : 
$$\phi = \underset{\Sigma}\oiint \overrightarrow{E}.\overrightarrow{n}.dS = \dfrac{Q_{int}}{\varepsilon_0}$$
\end{enon}
\begin{prop}
D'après le théorème de Gauss, on peut déduire que le potentiel ne peut pas être extremum au niveau d'un point vide de charge.
\end{prop}
\subsection{Forme locale du théorème de Gauss}
\subsubsection{Opérateur Divergence}
\begin{de}
On défini l'opérateur divergence, notée div, par : 
$$div : \Re^3 \rightarrow \Re$$
$$\overrightarrow{D} \mapsto div(\overrightarrow{D})$$
Supposons que $\overrightarrow{D}(x,y,z)$, alors :
$$div(\overrightarrow{D}) = \dfrac{\partial D_x}{\partial x} + \dfrac{\partial D_y}{\partial y} + \dfrac{\partial D_z}{\partial z}$$
\end{de}
\begin{prop}
Considérons une surface fermé $\Sigma$, de volume intérieur chargé V : 
$$\phi = \underset{\Sigma}\iint\overrightarrow{E}(M).\overrightarrow{n}.dS = \underset{V}\iiint div(\overrightarrow{E}).dV$$
\end{prop}
\subsection{Équation de Poisson}
\subsubsection{Opérateur Laplacien}
\begin{de}
On défini l'opérateur Laplacien, noté $\Delta$, par : 
$$\Delta : \Re \rightarrow \Re$$
$$A \mapsto \Delta(A)$$
Avec : 
$$\Delta(A) = \dfrac{\partial^2 A}{\partial^2 x} + \dfrac{\partial^2 A}{\partial^2 y} + \dfrac{\partial^2 A}{\partial^2 z}$$ 
\end{de}
\subsubsection{Énoncé}
Considérons une surface fermé, $\Sigma$, de densité volumique de charge $\rho$, de volume chargé V.\\
D'après le théorème de Gauss, on obtient que : 
$$\phi = \underset{\Sigma}\oiint\overrightarrow{E}(M).\overrightarrow{n}.dS = \underset{V}\iiint \dfrac{\rho.dV}{\varepsilon_0} = \underset{V}\iiint div(\overrightarrow{E}).dV$$
On en déduit donc le relation locale suivante, qui dépend du point M :
$$\dfrac{\rho}{\varepsilon_0} = div(\overrightarrow{E})$$
Or, par définition : 
$$\overrightarrow{E} = -\overrightarrow{grad}(V)$$
On obtient donc que : 
$$div(\overrightarrow{E}) = -\Delta(V)$$
Avec $\Delta(V)$ le Laplacien de V.\\
On obtient donc que : 
$$\dfrac{\rho}{\varepsilon_0} = -\Delta V$$
Ceci constitue l'équation de poisson.
