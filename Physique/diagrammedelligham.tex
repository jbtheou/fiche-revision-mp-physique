
\chapter{Diagramme d'Ellingham}
\section{Construction du diagramme}
\subsection{Oxyde}
\begin{de}
Un oxyde est de la forme $M_yO_x$, avec M un élément (ici un métal) , et O l'oxygène.\\
En général, le nombre d'oxydation de l'oxygène est -II.\\
Si ce n'est pas le cas, on appele ces entités des peroxyde.\\
On defini aussi des oxydes mixtes, très présent dans les roches par exemple. Exemple : 
$$Al_2N_gO_4$$
Un oxyde peut passer par les trois états de la matière, mais on le trouve majoritairement, sous les "conditions normal" de température et de pression, dans l'état solide.
\end{de}
\subsubsection{Structure}
La structure d'un oxyde peut être des liaisions covalente, ou une forme parfaitement ionique, ou encore un mixte de ces deux structures.
\subsubsection{Propriétés acide-base}
Les oxydes peuvent etre acide au sens de Brönsted, c'est à dire capable de céder un ou plusieurs $H^+$, comme au sens de Lewis, c'est a dire capable de capter des doublets.\\
Il peuvent aussi etre des bases au sens de Brönsted, c'est a dire capable d'accepter un ou plusieurs $H^+$.
\subsubsection{Réaction de formation d'un oxyde}
Par convention, dans toutes réactions de formation d'un oxyde, on pose 1 pour le coefficiant stochiométrique de $O_2$, et on équilibre l'équation en conséquence.
\subsection{L'approximation d'Elligham}
Dans les réactions étudié, à savoir les réactions de formation des métaux à partir des oxydes, on peut introduire l'enthalpie libre :
$$\Delta G^r (T) = \Delta H^r (T) - T.\Delta S^r(T)$$
L'approximation d'Elligham consiste à dire qu'en l'absence de changement d'état, l'ethalpie de formation et l'entropie de formation sont des constantes indépente de la température. On obtient donc une fonction affine en fonction de la température pour l'enthalpie libre.\\
Si il y a un changement d'état, la fonction reste affine, mais la pente est différente, il y a donc une brisure de pente.
\section{Utilisation du Diagramme d'Elligham}
\subsection{Prévison des réactions}
On utilise le diagramme d'Elligham pour étudier la faisabilité d'une réaction.\\
En effet, la réaction à lieu si l'affinité chimique A est positive, donc si $\Delta G^r(T)$ est négatif, et on peut lire cette information directement sur le diagramme d'Elligham.\\
Lorsque les deux droites impliqué dans la réaction se croise, on appele la température où ceci arrive température d'inversion. Cette température correspond a : 
$$\Delta G^r(T)=0$$
\subsection{Domaine d'existence}
Grâce au diagramme d'Elligham, on obtient le domaine d'existence du métal et de son oxyde, à un T fixé.
