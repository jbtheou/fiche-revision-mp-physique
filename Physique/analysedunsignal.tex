\chapter{Analyse d'un signal}
\section{Signal périodique}
Soit f une fonction.
\begin{de}
f est une fonction périodique si :
$$\exists T~ tq~ \forall t~ f(t+T) = f(t)$$
\end{de}
\subsection{Caractéristique principale}
Soit f une fonction périodique.
\subsubsection{Période, fréquence}
Ces deux caractéristiques sont fondamentale. La fréquence de la fonction est l'inverse de la période :
$$f = \dfrac{1}{T}$$
\subsubsection{Valeur moyenne}
\begin{de}
On défini la valeur moyenne d'une fonction f de la façon suivante : 
$$\left\langle f(t) \right\rangle = \dfrac{1}{T} \int_{t_0}^{t_0+T} f(t)dt $$
\end{de}
\subsubsection{Valeur efficace}
\begin{de}
On défini la valeur efficace d'une fonction f de la façon suivante :
$$F_e = \sqrt{\left\langle f^2(t) \right\rangle}$$
On peut l'écrire aussi sous la forme : 
$$F_e^2 = \dfrac{1}{T} \int_{t_0}^{t_0+T} f^2(t)dt$$
\end{de}
\section{Décomposition d'un signal périodique en fonction sinusoïdale}
\subsection{Théorème de Fourier}
\begin{enon}
Considérons une fonction périodique f : 
$$\exists T~ tq~ \forall t~ f(t+T) = f(t)$$
Il existe deux suites, ($a_n$) et $(b_n)$, telque : 
$$f(t) = \dfrac{a_0}{2} + \sum_{n=1}^{\infty} a_n\cos\left(\dfrac{2\pi n t}{T}\right) + \sum_{m=1}^{\infty} b_m\sin\left(\dfrac{2\pi m t}{T}\right)$$
\end{enon}
Les suites $(a_n)$ et $(b_n)$ sont appelé coefficients de Fourier.
\begin{enon}
$\forall n \in N$ : 
$$a_n = \dfrac{2}{T}\int_{t_0}^{t_0+T} f(t)\cos\left(\dfrac{2\pi nt}{T}\right)dt$$
$$b_n = \dfrac{2}{T}\int_{t_0}^{t_0+T} f(t)\sin\left(\dfrac{2\pi nt}{T}\right)dt$$
\end{enon}
\begin{prop}
On observe que : 
\begin{itemize}
 \item[$\rightarrow$] Si f est impaire, $\forall n \in N~ a_n=0$
 \item[$\rightarrow$] Si f est paire, $\forall n \in N~ b_n=0$
\end{itemize}
\end{prop}
\subsubsection{Coefficients de Fourier complexe}
\begin{prop}
En passant $\cos$ et $\sin$ en complexe, dans les coefficients de Fourier, on obtient que : 
$$f(t) = \sum_{-\infty}^{+\infty}c_n.e^{\frac{i2\pi nt}{T}}$$
avec : 
$$c_n = \dfrac{1}{T}\int_{t_0}^{t_0+T} f(t)e^{\frac{-i2\pi nt}{T}}dt$$
\end{prop}
\subsection{Identité de Parseval}
Soit f fonction périodique.\\
\begin{de}
Considérons le développement en série de Fourier de f : 
$$f(t) = \dfrac{a_0}{2} + \sum_{n=1}^{\infty} a_n\cos\left(\dfrac{2\pi n t}{T}\right) + \sum_{m=1}^{\infty} b_m\sin\left(\dfrac{2\pi m t}{T}\right)$$
On en déduit que : 
$$\left\langle f^2(t)\right\rangle = \dfrac{a_0}{4} + \sum_1^{\infty} \dfrac{a^2_n}{2} + \sum_1^{\infty} \dfrac{b²_n}{2}$$
car tous les doubles produits, issue de l'élévation au carré, qui comporte des fonctions cos ou sin ont une valeur moyenne nulle.
En posant :
$$c_n = \dfrac{a_n - ib_n}{2}$$
On obtient : 
$$\left\langle f^2(t)\right\rangle = \sum_{-\infty}^{+\infty} |c_n|^2$$
 
\end{de}

\subsection{Transformée de Fourier}
\begin{de}
Considérons f, une fonction.\\
On associe à f une fonction périodique $\overset{\sim}f$ : 
$$\forall t~ f(t) \rightarrow \overset{\sim}f(\nu)$$
Avec : 
$$\overset{\sim}f(\nu) = \int_{-\infty}^{+\infty} f(t)e^{-i2\pi\nu t}dt$$
On peut retourner cette égalité.
\end{de}

