\chapter{Régime quasi-stationnaire}
\section{Approximation des régimes quasi-stationnaire}
\subsection{Définitions}
Nous avons les relations suivantes :\\ 
\begin{itemize}
 \item[$\rightarrow$] $\Delta(V) - \dfrac{1}{c^2}.\dfrac{\delta^2 V}{\delta t^2} = \dfrac{-\rho}{\varepsilon_0}$\\
 \item[$\rightarrow$] $\overrightarrow{\Delta(\overrightarrow{A})} - \dfrac{1}{c^2}.\dfrac{\partial^2 \overrightarrow{A}}{\partial t^2} = -\mu_0.\overrightarrow{j}$\\
\end{itemize}
On montre que l'on peut faire l'approximation des régimes quasi-stationnaires si : 
$$d \ll c.\tau$$
Avec d une distance caratéristique et $\tau$ un temps caractéristique.
\subsection{Approximation du régime quasi-stationnaire électrique}
On a :
$$\overrightarrow{E} = - \overrightarrow{grad}(V) - \dfrac{\partial \overrightarrow{A}}{\partial t}$$
On fait l'approximation du régime quasi-stationnaire électrique si : 
$$\parallel \overrightarrow{j} \parallel \ll \rho.c$$
Dans ce cas, on peut écrire : 
$$\overrightarrow{E} = - \overrightarrow{grad}(V)$$
Ceci modifie les équations de Maxwell, car on obtient :
$$\overrightarrow{rot}(\overrightarrow{E}) = \overrightarrow{0}$$
\subsection{Approximation du régime quasi-stationnaire magnétique}
On a : 
$$\overrightarrow{rot}(\overrightarrow{B}) = \mu_0.\overrightarrow{j}+\varepsilon_0.\mu_0.\dfrac{\partial \overrightarrow{E}}{\partial t}$$
On fait l'approximation du régime quasi-stationnaire magnétique, si : 
$$\parallel \varepsilon_0.\mu_0.\dfrac{\partial \overrightarrow{E}}{\partial t} \parallel \ll \parallel \overrightarrow{j}\parallel$$
Dans ce cas, on obtient donc que : 
$$\overrightarrow{rot}(\overrightarrow{B}) = \mu_0.\overrightarrow{j}$$
\section{Cas des conducteurs ohmique}
\subsection{Loi d'ohm}
Un conducteur vérifie la loi d'ohm si :
$$\overrightarrow{j} = \gamma.\overrightarrow{E}$$
$$\overrightarrow{j} = \rho_m.<\overrightarrow{v}>$$
Avec $<\overrightarrow{v}>$ la valeur moyenne de la vitesse et $\rho_m$ la densité de charge mobile.
\subsection{Densité volumique de charge}
La conservation de la charge est donnée par la relation : 
$$div(\overrightarrow{j}) + \dfrac{\partial \rho}{\partial t} = 0$$
On montre que la densité de charge peut se mettre sous la forme : 
$$\rho = \rho_0.e^{-\dfrac{\gamma}{\varepsilon_0}.t}$$
On montre que la densité de charge devient nul en un temps extremement court, de l'ordre de $10^{-11}$ pour le cuivre par exemple.
\subsection{Cas du régime sinusoïdale}
Dans le cas sinosoïdale, on montre qu'on peut considérer qu'on est en régime quasi stationnaire si : 
$$f \ll \dfrac{1}{2.\Pi.\tau}$$
\subsection{Interface conducteur-vide}
\subsubsection{Densité de courant}
Dans le cas d'un fil, dans un régime statique, le vecteur densité $\overrightarrow{j}$ à la surface du fil est tangent, car autrement, il y aurait accumulation de charge.\\
Dans le cas du régime variable, on montre que $\overrightarrow{j}$ n'est pas forcement tangent à la surface.
\subsection{Cas du conducteur parfait}
\subsubsection{Définitions}
On dit d'un conducteur qu'il est parfait si :
$$\gamma \rightarrow + \infty$$
Avec :
$$\overrightarrow{j} = \gamma.\overrightarrow{E}$$
\subsubsection{Champs électrique à l'interieur du conducteur}
On montre que dans un conducteur, le courant passe à la surface. Ceci est appelé effet de peau
