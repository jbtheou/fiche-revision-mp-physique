\chapter{corrosion}
\section{Réaction d'oxydoréduction d'un métal}
\begin{de}
La corrosion d'un métal est l'oxydation de cet élement à l'état d'ion métallique. En milieu aqueux, cette réaction est appelé corrosion humide.
\end{de}
Cette réaction d'oxydation du métal à besoin d'un oxydant, susceptible de capter les électrons produit. On obtient donc que la corrosion du métal M a pour équation :
$$M + ox \rightarrow M^{n+} + red$$
\subsection{Réaction sur une électrode}
\subsubsection{Description de l'interface métal-solution}
Dans le cas d'un morceau en solution, on obtient qu'il peut exister une différence de potentiel entre la solution et le morceaux de métal, contrairement au cas de deux solutions.
\subsubsection{Réaction}
Le système est composé de deux électrodes. L'électrode où à lieu la réaction de réduction est appelé cathode. C'est dans cette électrode qu'arrive les électrons. L'électrode où à lieu la réaction d'oxydation est appelé anode, c'est de cette électrode que parte les électrons.
En résumé "imagé", nous avons : 
\begin{itemize}
 \item[$\rightarrow$] La cathode : Écurie des cations
 \item[$\rightarrow$] L'anode : Écurie des anions
\end{itemize}
\section{Cinétique}
\subsection{Intensité de corrosion}
Considérons l'anode. Sur l'anode, nous avons la réaction suivante : 
$$ox + z.e^- \rightarrow red$$
La charge élementaire est donc : 
$$dq = z.d\varepsilon.Na.e$$
Avec $\varepsilon$ l'avancement.
On obtient donc, sachant que : 
$$i = \frac{dq}{dt}$$
Que : 
$$i = z.Na.e.v$$
On peut donc déterminer, juste en mesurant l'intensité, la vitesse de la réaction v.
\subsection{Courbe intensité-potentiel}
\subsubsection{Courant anodique, courant cathodique}
\subsubsection{Courant anodique}
Dans le cas ou i>0, nous sommes donc quand le cas du courant anodique. On montre que l'on peut écrire, en supposant que la réaction admet un ordre : 
$$i_A = k.[Ox]$$
Ceci est vrai, sachant que nous avons établi une relation liant i à v. De plus, nous avons la variation de k avec la température d'après la loi d'Arrhenius : 
$$k = k_0.e^{\dfrac{-E_{m,a}}{R.T}}$$
\subsubsection{Courant cathodique}
De même, à la cathode, nous avons le courant cathodique, défini par : 
$$i_c = - k'.[red]$$
Avec toujours : 
$$k' = k'_0.e^{\dfrac{-E_{m,a}}{R.T}}$$
\subsubsection{Équilibre}
A l'équilibre, il n'y à pas de courant dans le circuit. Nous avons donc les relations suivantes à l'équilibre : \begin{itemize}
 \item[$\rightarrow$] $i_T = i_c + i_a = 0$
 \item[$\rightarrow$] $\mu_{ox} = \mu_{red}$
\end{itemize}
En développant la première condition, on obtient qu'elle implique que : 
$$\dfrac{[ox]}{[red]}=\dfrac{k_{0c}}{k_{0a}}$$
On obtient donc que le potentiel à une valeur bien définie par $\dfrac{k_{0c}}{k_{0a}}$ à l'équilibre.
\subsubsection{Loi de Bothlet-Volmer}
Par définition, nous avons :
$$i = i_a + i_c$$
En développant les énergies d'activation dans la formule d'Arrhenius, on obtient que : 
$$i = i_0[e^{-\dfrac{\Delta G_a^* + \alpha.\Delta E}{RT}} - e^{-\dfrac{\Delta G_c^* - (1 - \alpha).\Delta E}{RT}}]$$
On peut, à partir de cette relation, définir les systèmes lents et les systèmes rapides : 
\begin{itemize}
 \item[$\rightarrow$] Si $i_0$ est grand, on obtient que le système est rapide.
 \item[$\rightarrow$] Si $i_0$ est petit, on obtient que le système est lent.
\end{itemize}
\subsubsection{Courant de diffusions}
On montre que le profils intensité-potentiel est borné en intensité. Il admet des intensités limité, appelé intensité de diffusion. Ces limites sont du au fait que le système tend à s'homogénéiser.
