\chapter{Équations de Maxwell}
\section{Les quatres équations}
\subsection{Flux magnétique}
L'équation de Maxwell du flux magnétique, est donnée par : 
$$div(\overrightarrow{B}) = \overrightarrow{0}$$
Ceci est la forme locale de l'équation. Nous avons aussi une forme intégrale, à l'aide du théorème de Stockes : 
$$\underset{\Sigma}\iint \overrightarrow{B}.\overrightarrow{n}.dS = 0$$
Avec $\Sigma$ une surface fermé.
\subsection{Équation de Maxwell-Faraday}
L'équation de Maxwell-Faraday s'écrit, sous sa forme locale :
$$\overrightarrow{rot}(\overrightarrow{E}) = \dfrac{- \partial \overrightarrow{B}}{\partial t}$$
En faisant circuler le champs électrique, on obtient que : 
$$e = \dfrac{-d \Phi}{dt}$$
Ceci est l'expression globale. On remarque que les deux équations précédentes sont idépendante du milieu considéré.
\subsection{Équation de Maxwell-Gauss}
La forme locale est donnée par :
$$div(\overrightarrow{E}) = \dfrac{\rho}{\varepsilon_0}$$
On en déduit la forme global, qui n'est d'autre que le théorème de Gauss : 
$$\underset{\Sigma}\iint \overrightarrow{E}.\overrightarrow{n}.dS = \dfrac{Q_{int}}{\varepsilon_0}$$
\subsection{Équation de Maxwell-Ampère}
La forme locale de cette équation est donnée par :
$$\overrightarrow{rot}(\overrightarrow{B}) = \mu_0.\overrightarrow{j} + \varepsilon_0.\mu_0\dfrac{\partial \overrightarrow{E}}{\partial t}$$
On observe donc que cette forme locale est la forme locale du théorème d'Ampère, à laquelle on à ajouté une différentielle du champs éléctrique. Cette modification était obligatoire pour corriger des abérrations du théorème d'Ampère.
\section{Propriétés des champs}
\subsection{Conditions limites}
\subsubsection{Du champs électrique}
Considérons une surface de densité surfacique $\sigma$, séparant deux milieux (1) et (2). On obtient que : 
$$\overrightarrow{E_2} - \overrightarrow{E_1} = \dfrac{\sigma}{\varepsilon_0}\overrightarrow{n_{1\rightarrow 2}}$$
Avec $\overrightarrow{E_k}$ le champs électrique crée dans le milieu k, en un point très prés de la surface de séparation.
\subsubsection{Champs magnétique}
Considérons un "cylindre élémentaire", parcouru par un vecteur densité de courant $\overrightarrow{j_s}$. On obtient que : 
$$\overrightarrow{B_2}-\overrightarrow{B_1} = \mu_0.\overrightarrow{j_s}\wedge\overrightarrow{n_{1\rightarrow 2}}$$
Avec $\overrightarrow{B_k}$ le champs magnétique crée dans le milieu k, en un point très prés de la surface de séparation.
\subsection{Équations de propagation des champs}
\subsubsection{Champs électrique}
En appliquant l'opérateur $\overrightarrow{rot}$ à $\overrightarrow{rot}(\overrightarrow{E})$, on obtient l'équation de propagation du champs électrique : 
$$\overrightarrow{\Delta(\overrightarrow{E})} - \varepsilon_0.\mu_0.\dfrac{\partial^2\overrightarrow{E}}{\partial t^2} = \mu_0\dfrac{\partial \overrightarrow{j}}{\partial t} + \overrightarrow{grad}(\dfrac{\rho}{\varepsilon_0})$$
\subsubsection{Champs magnétique}
De la même façon pour le champs magnétique, on obtient que :
$$\overrightarrow{\Delta(\overrightarrow{B})} - \varepsilon_0.\mu_0.\dfrac{\partial^2 \overrightarrow{B}}{\partial t^2} = \mu_0.\overrightarrow{rot}(\overrightarrow{j})$$
\section{Potentiels}
\subsection{Définitions}
\subsubsection{Potentiel vecteur}
En partant de la définition du potentiel vecteur $\overrightarrow{A}$, on montre que l'on peut obtenir un autre potentiel vecteur $\overrightarrow{A'}$ de la façon suivante
$$\overrightarrow{A'} = \overrightarrow{A} + \overrightarrow{grad}(\Phi)$$
\subsubsection{Potentiel scalaire V}
En partant de l'expression de $\overrightarrow{rot}(\overrightarrow{E})$, on montre que : 
$$\overrightarrow{E} = \dfrac{-\partial \overrightarrow{A}}{\partial t} - \overrightarrow{grad}(V)$$
On obtient donc l'expression connu dans le cas statique.\\
De même que pour le potentiel vecteur, on montre que : 
$$V' = V - \dfrac{\partial \Phi}{\partial t} + cte$$
$\Phi$ est donc le lien entre $\overrightarrow{A}$ et V. On ne peut pas fixer arbitrairement $\overrightarrow{A}$ et V, ces deux potentiels sont liée.
\subsection{Conditions de Jauge}
\subsubsection{Équations verifié par le potentiel vecteur et V}
On montre que ces deux potentiels doivent vérifier les relations suivantes : 
$$\overrightarrow{\Delta(\overrightarrow{A})} - \varepsilon_0.\mu_0.\dfrac{\partial^2 \overrightarrow{A}}{\partial t^2} = \mu_0.\overrightarrow{j} + \overrightarrow{grad}(div(\overrightarrow{A})+\varepsilon_0.\mu_0.\dfrac{\partial V}{\partial t})$$
\subsubsection{Jauge de Coulomb}
La jauge de Coulomb est de poser que : 
$$div(\overrightarrow{A}) = 0$$
L'équation précédente devient donc : 
$$\overrightarrow{\Delta(\overrightarrow{A})} - \varepsilon_0.\mu_0.\dfrac{\partial^2 \overrightarrow{A}}{\partial t^2} = \mu_0.\overrightarrow{j} + \varepsilon_0.\mu_0.\dfrac{\partial}{\partial t}\overrightarrow{grad}(V)$$
\subsubsection{Jauge de Lorentz}
La jauge de Lorentz est de poser que : 
$$div(\overrightarrow{A})+\varepsilon_0.\mu_0.\dfrac{\partial V}{\partial t} = 0$$
À l'aide des ces jauges, on obtient donc deux équations : 
$$\overrightarrow{\Delta(\overrightarrow{A})} + \epsilon_0.\mu_0.\dfrac{\partial^2 \overrightarrow{A}}{\partial t^2} = -\mu_0.\overrightarrow{j}$$
$$\Delta(V) + \epsilon_0.\mu_0.\dfrac{\partial^2 V}{\partial t^2} = -\dfrac{\rho}{\varepsilon_0}$$

