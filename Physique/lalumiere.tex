\chapter{La lumière}
\section{Modèle scalaire de la lumière}
\subsection{Amplitude lumineuse}
Soit M un point de l'espace, et t un instant. On associe à M et t : 
$$\varphi : (M,t) \mapsto \varphi(M,t)$$
Cette fonction $\varphi$ est appelé amplitude lumineuse. Cette grandeur est inobservable dans la réalité.\\
L'éclairement est défini comme la puissance lumineuse qui arrive sur une Surface. On le note E. On montre que : 
$$E \propto |\varphi|^2 $$
\subsection{Notion d'onde}
Dans ce chapitre, on considère deux types d'onde : 
\subsubsection{Onde plane monochromatique}
L'amplitude lumineuse associé à ce type d'onde est donnée par : 
$$\varphi(M,t) = A.cos(\omega.t - k.x = \theta)$$
On peut aussi adopter la notation complexe, ce que nous allons faire dans la suite
\subsubsection{Onde sphérique monochromatique}
L'amplitude lumineuse est ici donnée par : 
$$\varphi(M,t) = \dfrac{A}{r}.cos(\omega.t - k.x + \theta)$$
\subsection{Relation d'ondes - Rayon lumineux}
Le rayon optique, considéré en optique géométrique, peut être représenté comme un tube selon lequel se propage l'onde. Le rayon lumieux donne la direction de propagation de l'onde.
\subsubsection{Théorème de Malus-Dupin}
\begin{theo}
La surface d'onde, défini commme : 
$$\left\lbrace M / \varphi(M,t) = \varphi(M_0,t) \right\rbrace $$
C'est à dire, cette surface d'onde défini touts les points ayant la même amplitude lumieuse.\\
Nous considérons deux types de surfaces d'ondes : 
\begin{itemize}
 \item[$\rightarrow$] Pour une onde plane : x = constante
 \item[$\rightarrow$] Pour une onde sphérique : Une sphère de rayon r
\end{itemize}
On montre que les lentilles ont pour effet de modifier cette surface d'onde. La modification entrainé dépend de la lentille considéré.
\end{theo}
\begin{theo}
Soit S une source et S' l'image de cette source par une lentille. Le temps mis pour relier S et S' est indépendent du chemin suivit.
\end{theo}
À l'aide de ce dernier théorème, nous pouvons redémontrer les lois de Snell-Descartes.
\section{Vision}
\subsection{Formation des images}
\subsubsection{Un appareil d'optique}
L'oeil, qui permet la formation d'image, est un appareil optique, qui n'est pas assimilable à une lentille simple. On montre que l'oeil s'accomode aussi, pour pouvoir être toujours au point.
\subsubsection{Un capteur : La rétine}
La rétine est composé de deux entités :
\begin{itemize}
 \item[$\rightarrow$] Les batonnets : En périphérie de la rétine, en très grand nombre, très sensible
  \item[$\rightarrow$] Les cones : Au centre, en petit nombre mais très consentré spatialement. Responsable de la vision en couleur.
\end{itemize}
L'oeil bouge constament, car autrement, l'information n'est plus actualisé.
\subsubsection{Vision des couleurs}
L'oeil ne perçoit pas les longeurs d'ondes. On défini une couleur comme un contenu spectral, c'est à dire une énergie en fonction de la longeur d'ondes. Nous avons donc une sentation coloré. Cependant, avec la norme RGB, on ne peut pas restituer toutes les couleurs, car certain couleurs sont formé à l'aide de coefficiants négatifs pour R, G ou B.
\subsection{Sources de Lumière}
Nous avons trois types de sources de lumière : 
\begin{itemize}
 \item[$\rightarrow$] Les lasers : Source cohérente, c'est à dire que $\theta$, dans l'amplitude lumineuse, est parfaitement déterminé. Cette source est très proche d'une sources monochromatique, en effet sa composition en longeur d'onde est défini à l'aide d'une courbe de Gauss, avec à mi-hauteur un écart de l'ordre de 0,3 nm.
 \item[$\rightarrow$] Les lampes spectrales : Source incohérente, car, de par l'agitation thermique, nous avons, par effet Doppler-Fiseau, un eventail de longeur d'onde, et non une longeur d'onde. De plus, la lumière étant émise par les atomes qui se dé-excite, il ne sont pas syncrone, donc $\theta$ est totalement aléatoire, dans l'amplitude lumineuse.
 \item[$\rightarrow$]
\end{itemize}