\chapter{\'Energie Electrostatique}
\section{\'Energie d'une charge dans un champs électrique exterieur}
Soit M(q) une charge. On peut écrire l'énergie potentielle d'interaction de la charge q avec le champ exterieur sous la forme : 
$$E_p = q.V(M)$$
Avec V(M) le potentiel crée par les charges exterieurs.
\section{\'Energie électrique d'un système de charges fixes}
Soit A($q_A$) et B($q_B$) deux charges. On montre que l'on peut écrire l'énergie potentielle de deux charges sous la forme : 
$$E_p = \dfrac{q_A.q_B}{4\pi\varepsilon_0.r_{ab}}$$
Avec $r_{ab}$ la distance entre les charges A et B.
\subsection{Cas d'un ensemble de N charges}
Considérons le cas de n charges. On peut écrire l'énergie potentielle sous la forme : 
$$E_p = \dfrac{1}{2}\sum_i q_i.V_i$$
En notant $V_i$ le potentiel électrostatique crée par les charges autres que i au point $M_i$, occupé par la charge $q_i$.
\subsection{Généralisation}
On peut généraliser ceci à une distribution continue de charges, en écrivant l'énergie potentielle d'interaction sous la forme : 
$$E_p = \dfrac{1}{2}.\iiint \rho.V.d\tau$$
Avec $d\tau$ un élément de volume et et V le potentiel.
\section{Densité d'énergie électrostatique}
En partant de l'expression de l'énergie potentielle sous forme intégrale, et d'après le théorème de Gauss, on obtient que : 
$$E_p = \dfrac{1}{2}.\iiint \rho.V.d\tau = \dfrac{1}{2}.\iiint \varepsilon_0.E^2.d\tau$$
Il faut bien notée que le domaine d'integration n'est pas le même. Pour la première intégrale, on intègre sur l'espace où $\rho \neq 0$. Pour la seconde, on intègre sur l'espace tout entier. On peut donc introduire une densité volumique d'énergie, donnée par : 
$$\dfrac{1}{2}.\varepsilon_0.E^2$$
\section{\'Energie Magnétique et Électromagnétique}
\subsection{Densité volumique d'énergie magnétique}
De façon analogue à précédement, on montre que : 
$$E =\iiint \dfrac{B^2}{2.\mu_0}dv$$
On montre bien que l'énergie est réparti de façon uniforme dans tout l'espace, avec un densité volumique de :
$$\dfrac{B^2}{2.\mu_0}$$
\subsection{Bilan d'énergie dans un volume élémentaire}
On postule l'existence d'une densité volumique d'énergie u telle que l'énergie électromagnétique d'un système puisse s'écrire :
$$\overrightarrow{E} = \iiint \overrightarrow{u}.dv$$
De même, on postule qu'il existe un vecteur densité de flux de puissance, noté $\Pi$, telque : 
$$\overrightarrow{P} = \iint \Pi.\overrightarrow{n}.dS$$
On montre que : 
$$\dfrac{\partial u}{\partial t} + div(\overrightarrow{\Pi}) + \overrightarrow{J}.\overrightarrow{E} = 0$$
Ceci constitue le bilan d'énergie dans un volume élémentaire.
\subsection{Vecteur de Poynting}
On montre que $\overrightarrow{\Pi}$, appelé vecteur de Poynting, a pour expression : 
$$\overrightarrow{\Pi} = \dfrac{\overrightarrow{E}\wedge\overrightarrow{B}}{\mu_0}$$
On montre que ce vecteur donne la direction dans lequel se transfert l'énergie.
\section{L'essentiel}
Nous devons retenir les choses suivantes : 
\subsection{\'Equation de Maxwell}
\begin{itemize}
 \item[$\rightarrow$] $div(\overrightarrow{B}) = 0$
 \item[$\rightarrow$] $\overrightarrow{rot}(\overrightarrow{E}) = \dfrac{-\partial\overrightarrow{B}}{\partial t}$
 \item[$\rightarrow$] $div(\overrightarrow{E}) = \dfrac{\rho}{\varepsilon_0}$
 \item[$\rightarrow$] $\overrightarrow{rot}(\overrightarrow{B}) = \mu_0.\overrightarrow{j}+\varepsilon_0.\mu_0.\dfrac{\partial \overrightarrow{E}}{\partial t}$
\end{itemize}
\subsection{Force de Lorentz}
Une charge q plongé dans un champs $(\overrightarrow{E},\overrightarrow{B})$ subit la force suivante : 
$$\overrightarrow{f} = q.(\overrightarrow{E} + \overrightarrow{v}\wedge\overrightarrow{B})$$
\subsection{Autres expressions}
Les deux relations suivantes : 
\begin{itemize}
 \item[$\rightarrow$] $u = \dfrac{\varepsilon_0.E^2}{2} + \dfrac{B^2}{2.\mu_0}$
 \item[$\rightarrow$] $\overrightarrow{\Pi} = \dfrac{\overrightarrow{E}\wedge\overrightarrow{B}}{\mu_0}$ 
\end{itemize}
Avec toutes ces expressions, on redémontre l'ensemble des lois de l'électromagnétisme.